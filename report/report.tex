\documentclass[10pt, a4paper,spanish]{article}
\usepackage[utf8]{inputenc}

\usepackage{hyperref}

\usepackage[T1]{fontenc}

\usepackage[hmarginratio=1:1,top=32mm,columnsep=20pt]{geometry}
\usepackage[hang, small,labelfont=bf,up,textfont=it,up]{caption}


\usepackage{graphicx}
\graphicspath{ {images/} }

\usepackage{abstract}
\renewcommand{\abstractnamefont}{\normalfont\bfseries}
\renewcommand{\abstracttextfont}{\normalfont\small\itshape}

\usepackage{titlesec}
\renewcommand\thesection{\Roman{section}}
\renewcommand\thesubsection{\Roman{subsection}}
\titleformat{\section}[block]{\large\scshape\centering}{\thesection.}{1em}{}
\titleformat{\subsection}[block]{\large}{\thesubsection.}{1em}{}


\usepackage{fancyhdr}
\pagestyle{fancy}
\fancyhead{}
\fancyfoot{}
\fancyhead[C]{ \today \ $\bullet$ Minería de Datos $\bullet$ Multicomparación de Clasificadores}
\fancyfoot[RO]{\thepage}

%-------------------------------------------------------------------------------
%	TITLE SECTION
%-------------------------------------------------------------------------------

\title{\vspace{-15mm}\fontsize{24pt}{10pt}\selectfont\textbf{Multicomparación de \\ Clasificadores}} % Article title

\author{Sergio García Prado}
\date{\today}

%-------------------------------------------------------------------------------

\begin{document}

	\maketitle % Insert title

	\thispagestyle{fancy} % All pages have headers and footers

%-------------------------------------------------------------------------------
%	ABSTRACT
%-------------------------------------------------------------------------------

	\begin{abstract}
		\noindent
	\end{abstract}

%-------------------------------------------------------------------------------
%	TEXT
%-------------------------------------------------------------------------------

	\section{Introducción}

        \paragraph{}
		La comparación consistirá en dos partes principales: la primera de ellas se basa en un Test de Signos sobre 2 de los clasificadores para todos los conjuntos de datos, mientras que la segunda parte se refiere a la realización de un Ranking en el cuál participarán todos los clasificadores.

		\paragraph{}
		Para la realización de estas pruebas se ha utilizado Weka, que es una plataforma de software para el aprendizaje automático y la minería de datos escrita en Java, desarrollada en la Universidad de Waikato y distribuida como Software Libre.

		\paragraph{}
		Por lo tanto, lo primero es describir tanto los clasificadores como los conjuntos de datos que se utilizarán en los tests de clasificación:

		\subsection{Clasificadores}

			\begin{itemize}
				\item \textbf{SVM con kernel lineal}:
				\item \textbf{3-NN}:
				\item \textbf{Naive Bayes}:
				\item \textbf{J48}:
			\end{itemize}

		\subsection{Conjuntos de Datos}

			\begin{itemize}
				\item \textbf{Arrhythmia}:
				\item \textbf{Diabetes}:
				\item \textbf{Glass}:
				\item \textbf{Ionosphere}:
				\item \textbf{Iris}:
				\item \textbf{Labor}:
				\item \textbf{Seeds}:
				\item \textbf{Segment Test}:
				\item \textbf{Soybean}:
				\item \textbf{Vote}:
			\end{itemize}


	\section{Test de Signos: SVM y J48}

        \paragraph{}


	\section{Ranking}

        \paragraph{}



\end{document}
